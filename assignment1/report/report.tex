\documentclass[a4paper, 10pt, oneside, article]{memoir}
\chapterstyle{culver}
\checkandfixthelayout

\usepackage{lipsum}

% Palatino font
\usepackage{palatino}

% Font and input encoding
\usepackage[T1]{fontenc}
\usepackage[utf8x]{inputenc}

% Babel (language)
\usepackage[english]{babel}

% Support for blackboard bold symbols.
\usepackage{bbm}

% AMS-Math packages
\usepackage{amsmath}
\usepackage{amssymb}
\usepackage{amsthm}

% For including bitmap graphics
\usepackage{graphicx}

\usepackage{multirow}

\usepackage{todonotes}

% Remove chapters from figure counters
%\counterwithout{figure}{chapter}
% Add sections instead
%\counterwithin{figure}{section}
%\counterwithout{section}{chapter}

\usepackage{tikz}
\usetikzlibrary{shapes,arrows}

\title{Advanced Algorithms and Data Structures\\Max-flow/linear programming}
\author{Jens Raaby, Mads Hartmann \& Ulrik Rasmussen}

\begin{document}

\maketitle

\section*{Question 1}

The problem can easily be cast to an instance of max-flow.

Given is an undirected graph $G = (V, E)$, where each edge $(u,v) \in
E$ has a weight $w(u,v)$. We have a set $\{0, ..., 5\} \subset V$ of
sources, and a single sink, $t=19$. To solve the problem as a max-flow
problem, we first have to convert it to the form of a \emph{flow
  network}, i.e. we must ensure that the graph is directed, has no
antiparallel edges, and that there is a single source and sink.

We construct a new graph $G' = (V', E')$ as follows:

\begin{itemize}
\item Add all vertices $v \in V$ to $V'$.
\item For each edge $(u,v) \in E$, we introduce an extra vertex $v'$
  to $V'$, and edges $(v,u), (u,v'), (v', u)$ to $E'$ with $c(v,u) =
  c(u,v') = c(v', u) = w(u,v)$.
\item Add a single \emph{supersource} $s$ to $V'$ and edges $(s,0),
  (s,1), ..., (s,5)$ to $E'$ with $c(s,0) = c(s, 1) = ... = c(s,5) =
  \infty$.
\end{itemize}

This will yield a valid flow network $G'$. When finding a max-flow
$f^\star$ in this network, we can choose between the
\textsc{Ford-Fulkerson}, running in $O(V |f^\star|)$ time, and
\textsc{Edmonds-Karp}, running in $O(VE^2)$ time. It is evident that
\textsc{Ford-Fulkerson} is only asymptotically more efficient when $VE
< |f^\star|$. In our case, the maximum incoming flow to the sink is
$110$, and hence is an upper bound for $|f^\star|$. For this
particular network, \textsc{Ford-Fulkerson} would therefore be the
most efficient method.


\section*{Question 2}

We simply use the method from Cormen section 29.2 for expressing a
maximum flow problem as a linear program:

\begin{align*}
  \text{maximize} && \sum_{v\in V'} f_{sv} \\
  \text{subject to} && f_{uv} \leq{}& c(u,v) & \text{for each $(u,v) \in E'$}, \\
                    && \sum_{v \in V'} f_{vu} ={}& \sum_{v \in V'} f_{uv} & \text{for each $u \in V' - \{s, t \}$}, \\
                    && f_{uv} \geq{}& 0 & \text{for each $(u, v) \in E'$}.
\end{align*}

where each variable $f_{uv}$ represents the value of $f(u,v)$ in the
resulting flow. Note that we omit incoming edges to the source in the
objective function, as the procedure described in Question 1 ensures
that there are never any incoming edges to the supersource. For each
$(u,v) \not\in E$, we fix the corresponding variable $f_{uv}$ to 0.


\section*{Question 3}

The example graph expressed both as an undirected graph and converted
to a flow network can be seen in Figure \ref{fig:example-graph}. The
linear program for the flow network takes on the following (normal)
form:

\begin{align*}
  \text{maximize}   &&& f_{sw} \\
  \text{subject to} &&& f_{wz} &\leq 60 \\
                    &&& f_{zt} &\leq 60 \\
                    &&& f_{tw} &\leq 60 \\
                    &&& f_{wx} &\leq 30 \\
                    &&& f_{xa} &\leq 30 \\
                    &&& f_{aw} &\leq 30 \\
                    &&& f_{ay} &\leq 50 \\
                    &&& f_{yt} &\leq 50 \\
                    &&& f_{ta} &\leq 50 \\
                    &&& f_{ta} + f_{xa}  - f_{aw} - f_{ay} & \leq 0 \\
                    &&& -f_{ta} - f_{xa}  + f_{aw} + f_{ay} & \leq 0 \\
                    &&& f_{wx} - f_{xa} &\leq 0 \\
                    &&& -f_{wx} + f_{xa} &\leq 0 \\
                    &&& f_{ay} - f_{yt} &\leq 0 \\
                    &&& -f_{ay} + f_{yt} &\leq 0 \\
                    &&& f_{wz} - f_{zt} &\leq 0 \\
                    &&& -f_{wz} + f_{zt} &\leq 0 \\
                    &&& f_{wz} + f_{wx} - f_{tw} - f_{sw} & \leq 0 \\
                    &&& -f_{wz} - f_{wx} + f_{tw} + f_{sw} & \leq 0 \\
                    &&& f_{sw}, f_{wz}, f_{zt}, f_{tw}, f_{wx}, f_{xa}, f_{aw}, f_{ay}, f_{yt}, f_{ta} &\geq 0
\end{align*}

The solution to the above linear program is
$$(f_{sw}, f_{wz}, f_{zt}, f_{tw}, f_{wx}, f_{xa}, f_{aw}, f_{ay}, f_{yt}, f_{ta}) = (90, 60, 60, 0, 30, 30, 0, 30, 30, 0).$$

\begin{figure}[h]
  \centering
  \includegraphics[scale=0.75]{graph1.pdf}
  \includegraphics[scale=0.75]{graph1-flow.pdf}
  \caption{Example graph, before and after conversion to a valid flow
    network.}
  \label{fig:example-graph}
\end{figure}

\section*{Question 4}

Given a formulation of the computer network problem as a flow graph
$G=(V,E)$, we need to find an edge $(u,v) \in E$ for which increasing
the capacity $c(u,v)$ by one will lead to an increase in
$|f^\star|$. Moreover, the edge that is increased must be the edge
with the smallest capacity among all edges which can increase the
maximum flow.

An efficient method for finding such an edge is to first find the
maximum flow $f^\star$, and then compute the residual graph
$G_{f^\star}$, and hence finding the minimum cut $(S,T)$. We then
identify the following set of candidate edges
$$
 C = \left\{ (u,v) \in E : u \in S, v \in T,
\substack{\text{increasing $c(u,v)$ by one yields graph $G'$}
       \\ \text{with an augmenting path in $G'_{f^\star}$}} \right\}.
$$

We then pick the edge $(u,v) \in C$ with the smallest capacity, and
increase its capacity $c(u,v)$ by one. I.e. we define a new graph $G'$
with capacity $c'(a,b) = c(a,b)+1$ when $(a,b) = (u,v)$ and $c'(a,b) =
c(a,b)$ otherwise. Note that the set $C$ may be empty, which means
that there is no such single flow-increasing edge. Finding the max
flow and checking for augmenting paths can both be done in polynomial
time, so we regard this method as efficient.

We claim that if the set $C$ is not empty, the above method will yield
a new graph $G'$ with a max flow ${f^\star}'$ such that
\begin{enumerate}
\item $|{f^\star}'| > |f^\star|$, and that
\item there is no edge $(u',v') \in E$ where $c(u',v') < c(u,v)$ for
  which increasing $c(u',v')$ by one will yield another graph $G''$
  where $|{f^\star}''| > |f^\star|$.
\end{enumerate}

\begin{proof}
  Showing (1) is a direct consequence of Corollary 26.3. Since we
  chose $(u,v)$ such that increasing $c(u,v)$ would yield an
  augmenting path $p$ in $G'_{f^\star}$, ${f^\star}' = f^\star
  \uparrow f_p$ is a flow in $G'$, with $|{f^\star}'| > |f^\star|$.

  Showing (2) is a proof by contradiction. Assume that there is an
  edge $(u', v') \in E$ with $c(u',v') < c(u,v)$ where defining
  $c''(u', v') = c(u',v')+1$ yields a graph $G''$ with $|{f^\star}''|
  > |f^\star|$. This implies that $G_{f^\star}''$ contains an
  augmenting path $p$ going through $u'$ and $v'$, and specifically
  that there must be paths from $s$ to $u'$ and from $v'$ to $t$ in
  the original residual graph $G_{f^\star}$. Therefore, there cannot
  be a path from $s$ to $v'$ in $G_{f^\star}$, because then we would
  have an augmenting path in $G_{f^\star}$, a contradiction. This
  places $u' \in S$ and $v' \in T$, and hence also $(u', v') \in
  C$. Since we chose $(u,v) \in C$ with the minimum $c(u,v)$, we must
  have $c(u,v) \leq c(u', v')$. This contradicts our assumption that
  $c(u', v') < c(u,v)$, and hence we are done.
\end{proof}

\section*{Question 5}

We now write the dual of the linear program written in Question 3. We
will get one dual variable for each constraint in the original
program. The first $9$ contraints are concerned with the capacity on
edges, so we denote the corresponding dual variables by $y_{uv}$ for
edge $(u,v)\in E$. The next 5 pairs of constraints are concerned with
flow conservation for individual vertices, with two constraints per
vertex. We denote the corresponding pairs of dual variables by $z_v,
z'_v$ for vertices $v \in V$.

The resulting dual program can be seen below:

\begin{align*}
\text{minimize}   &&& 60y_{wz} + 60y_{zt} + 60y_{tw} + 30y_{wx} + 30y_{xa} + 30y_{aw} + 50y_{ay} + 50y_{yt} + 50y_{ta}  \\
\text{subject to} &&& z_{w}  -z'_{w} &\geq 1 \\
                  &&&y_{wz} + (z_{z}  -z'_{z})  - (z_{w} - z'_{w}) &\geq 0 \\
                  &&&y_{zt}  - (z_{z} - z'_{z}) &\geq 0 \\
                  &&&y_{tw} + (z_{w}  -z'_{w}) &\geq 0 \\
                  &&&y_{wx} + (z_{x}  -z'_{x})  -(z_{w} -  z'_{w}) &\geq 0 \\
                  &&&y_{xa} + (z_{a}  -z'_{a})  -(z_{x} - z'_{x}) &\geq 0 \\
                  &&&y_{aw}  -(z_{a} - z'_{a}) &\geq 0 \\
                  &&&y_{ay}  -(z_{a} - z'_{a}) + (z_{y}  -z'_{y}) &\geq 0 \\
                  &&&y_{yt}  -(z_{y} - z'_{y}) &\geq 0 \\
                  &&&y_{ta} + (z_{a}  -z'_{a}) &\geq 0 \\
                  &&& y_{wz}, y_{zt}, y_{tw}, y_{wx}, y_{xa}, y_{aw}, y_{ay}, y_{yt}, y_{ta}, z_{a}, z'_{a}, z_{x}, z'_{x}, z_{y}, z'_{y}, z_{z}, z'_{z}, z_{w}, z'_{w} &\geq 0
\end{align*}

We have solved the above program using the linear programming library
puLP. This yields an optimal solution where
\begin{equation*}
(v, y_{wz}, y_{wx}, z_{w}) =(90, 1, 1, 1),
\end{equation*}
where $v$ is the value of the objective function, and the rest of the
variables are zero.

\section*{Question 6}




\section*{Question 7}

\section*{Question 8}
The maximum bit-rate obtained for the given network was found to be 95. The minimum cost of leasing network from the external relays was  60 oere.

In order to find these values for the supplied network, we wrote a simple Python script using the puLP library as before.
First of all, the script imports a file describing the edge capacities (this was created from the supplied Java class). We created a graph class which can hold any object between 2 vertices. Two graphs are populated: one containing Linear Programming variables for each edge (specifying the lower and upper capacity bounds); the other graph contains the cost of sending a byte across that edge. We assumed that all edges have a cost of 0 apart from those which cross to or from one of the relay stations (edges with a connection to vertices 20-26), which have a cost of 1.
Since the network connections are bi-directional, an additional variable was created for each edge to allow antiparallel flow (this excludes flow back to the super source)
A single super-source was added with edges to the 5 workstations and no upper bound on their edges.

The 2 related problems were then defined in terms of their objective functions and other constraints. The objective function for maximum bitrate is simply maximising the flow through all edges leaving the supersource. The objective function for minimising the leasing price is minimising the cost per byte across each edge. Mathematically:
\begin{equation*}
\sum_{(u,v) \in E} a(u,v) f_{uv}
\end{equation*}
where $a(u,v)$ is the cost-per-byte across the edge from $u$ to $v$.

The remaining constraints were added to both problems. They share the flow conservation constraints, but the minimising cost problem has the extra constraint that the flow from the source must be the optimal flow for the network. This value was retrieved directly after solving the maximisation problem.

The flow for both problems happened to have the same solution, so running the minimal-cost-flow problem did not help reduce the cost.
\bibliographystyle{abbrv}
\bibliography{bibliography}
\end{document}
