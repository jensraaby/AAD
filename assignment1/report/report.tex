\documentclass[a4paper, 10pt, oneside, article]{memoir}
\chapterstyle{culver}
\checkandfixthelayout

\usepackage{lipsum}

% Palatino font
\usepackage{palatino}

% Font and input encoding
\usepackage[T1]{fontenc}
\usepackage[utf8x]{inputenc}

% Babel (language)
\usepackage[english]{babel}

% Support for blackboard bold symbols.
\usepackage{bbm}

% AMS-Math packages
\usepackage{amsmath}
\usepackage{amssymb}
\usepackage{amsthm}

% For including bitmap graphics
\usepackage{graphicx}

\usepackage{multirow}

\usepackage{todonotes}

% Remove chapters from figure counters
%\counterwithout{figure}{chapter}
% Add sections instead
%\counterwithin{figure}{section}
%\counterwithout{section}{chapter}

\usepackage{tikz}
\usetikzlibrary{shapes,arrows}

\title{Advanced Algorithms and Data Structures\\Max-flow/linear programming}
\author{Mads Hardmann \& Ulrik Rasmussen}

\begin{document}

\maketitle

\section*{Question 1}

The problem can easily be cast to an instance of max-flow.

Given is an undirected graph $G = (V, E)$, where each edge $(u,v) \in
E$ has a weight $w(u,v)$. We have a set $\{0, ..., 5\} \subset V$ of
sources, and a single sink, $t=19$. To solve the problem as a max-flow
problem, we first have to convert it to the form of a \emph{flow
  network}, i.e. we must ensure that the graph is directed, has no
antiparallel edges, and that there is a single source and sink.

We construct a new graph $G' = (V', E')$ as follows:

\begin{itemize}
\item Add all vertices $v \in V$ to $V'$.
\item For each edge $(u,v) \in E$, we introduce an extra vertex $v'$
  to $V'$, and edges $(v,u), (u,v'), (v', u)$ to $E'$ with $c(v,u) =
  c(u,v') = c(v', u) = w(u,v)$.
\item Add a single \emph{supersource} $s$ to $V'$ and edges $(s,0),
  (s,1), ..., (s,5)$ to $E'$ with $c(s,0) = c(s, 1) = ... = c(s,5) =
  \infty$.
\end{itemize}

This will yield a valid flow network $G'$. When finding a max-flow
$f^\star$ in this network, we can choose between the
\textsc{Ford-Fulkerson}, running in $O(V |f^\star|)$ time, and
\textsc{Edmonds-Karp}, running in $O(VE^2)$ time. It is evident that
\textsc{Ford-Fulkerson} is only asymptotically more efficient when $VE
< |f^\star|$. In our case, the maximum incoming flow to the sink is
$110$, and hence is an upper bound for $|f^\star|$. For this
particular network, \textsc{Ford-Fulkerson} would therefore be the
most efficient method.


\section*{Question 2}

We simply use the method from Cormen section 29.2 for expressing a
maximum flow problem as a linear program:

\begin{align*}
  \text{maximize} && \sum_{v\in V'} f_{sv} -{}& \sum_{v\in V'} f_{vs} \\
  \text{subject to} && f_{uv} \leq{}& c(u,v) & \text{for each $u,v\in V$}, \\
                    && \sum_{v \in V'} f_{vu} ={}& \sum_{v \in V'} f_{uv} & \text{for each $u \in V - \{s, t \}$}, \\
                    && f_{uv} \geq{}& 0 & \text{for each $u,v \in V$}.
\end{align*}


\section*{Question 3}

The example graph expressed both as an undirected graph and converted
to a flow network can be seen in Figure \ref{fig:example-graph}.

The solution to the linear program is
\begin{align*}
  f_{ab} = 20, ~~ f_{bd} = 25, ~~ f_{ac} = 15, ~~ f_{cb} = 5, ~~ f_{cd}=10,
\end{align*}
the rest of the variables are set to zero.


\bibliographystyle{abbrv}
\bibliography{bibliography}
\end{document}
